\documentclass[11pt,fleqn]{article}


%%%%%%%%%%%%%%%%%%%%%%%%%%%%%%%%%%%%%%%%%
%%% template that does not use Revtex4
%%% but allows special fonts
%%%%%%%%%%%%%%%%%%%%%%%%%%%%%%%%%%%%%%%%%


%%%%%%%%%%%%%%%%%%%%%%%%%%%%%%%%%%%%%%%%%%%
%%%
%%% Please use this template.
%%% Scroll down until you reach the \begin{document} part of the file
%%% Process the file in the Latex site 
%%%
%%%%%%%%%%%%%%%%%%%%%%%%%%%%%%%%%%%%%%%%%%%


% Page setup
\topmargin -1.5cm      
\oddsidemargin -0.04cm   
\evensidemargin -0.04cm  
\textwidth 16.59cm
\textheight 24cm 
\setlength{\parindent}{0cm} 
\setlength{\parskip}{0cm} 


% Fonts
\usepackage{latexsym}
\usepackage{amsmath} 
\usepackage{amssymb} 
\usepackage{bm}
\usepackage{graphicx}


% Math symbols I
\newcommand{\sinc}{\mbox{sinc}}
\newcommand{\const}{\mbox{const}}
\newcommand{\trc}{\mbox{trace}}
\newcommand{\intt}{\int\!\!\!\!\int }
\newcommand{\ointt}{\int\!\!\!\!\int\!\!\!\!\!\circ\ }
\newcommand{\ar}{\mathsf r}
\newcommand{\im}{\mbox{Im}}
\newcommand{\re}{\mbox{Re}}

% Math symbols II
\newcommand{\eexp}{\mbox{e}^}
\newcommand{\bra}{\left\langle}
\newcommand{\ket}{\right\rangle}

% Mass symbol
\newcommand{\mass}{\mathsf{m}} 
\newcommand{\Mass}{\mathsf{M}} 

% More math commands
\newcommand{\tbox}[1]{\mbox{\tiny #1}}
\newcommand{\bmsf}[1]{\bm{\mathsf{#1}}} 
\newcommand{\amatrix}[1]{\begin{matrix} #1 \end{matrix}} 
\newcommand{\pd}[2]{\frac{\partial #1}{\partial #2}}

% Other commands
\newcommand{\hide}[1]{}
\newcommand{\drawline}{\begin{picture}(500,1)\line(1,0){500}\end{picture}}
\newcommand{\bitem}{$\bullet$ \ \ \ }
\newcommand{\Cn}[1]{\begin{center} #1 \end{center}}
\newcommand{\mpg}[2][1.0\hsize]{\begin{minipage}[b]{#1}{#2}\end{minipage}}
\newcommand{\Dn}{\vspace*{3mm}}

% Figures
\newcommand{\putgraph}[2][0.30\hsize]{\includegraphics[width=#1]{#2}}

% heading
\newcommand{\exnumber}[1]{\newcommand{\exnum}{#1}}
\newcommand{\heading}[1]{\begin{center} {\Large {\bf Ex\exnum:} #1} \end{center}}
\newcommand{\auname}[1]{\begin{center} {\bf Submitted by:} #1 \end{center}}


\begin{document} %This is where you start editing things%
%%%%%%%%%%%%%%%%%%%%%%%%%%%%%%%%%%%%%%%%%%%%%%%%%%%%%%%%%%%%%%%%%%%%%%%%%%%%%%%%%%%%%%%%%%%%
%%%%%%%%%%%%%%%%%%%%%%%%%%%%%%%%%%%%%%%%%%%%%%%%%%%%%%%%%%%%%%%%%%%%%%%%%%%%%%%%%%%%%%%%%%%%
%%%%%%%%%%%%%%%%%%%%%%%%%%%%%%%%%%%%%%%%%%%%%%%%%%%%%%%%%%%%%%%%%%%%%%%%%%%%%%%%%%%%%%%%%%%%
%%%%%%%%%%%%%%%%%%%%%%%%%%%%%%%%%%%%%%%%%%%%%%%%%%%%%%%%%%%%%%%%%%%%%%%%%%%%%%%%%%%%%%%%%%%%

\exnumber{0909A3}
\heading{The generalized canonical transformation} 
\auname{Michael Pukshanski, Lior Weitzhandler}


{\bf The problem:}
\Dn
A system is described by the Hamiltonian H(q,p,t). The coordinate q is transformed q $\to$ Q=$\psi$(q,t)
\begin{enumerate}
\item Find the most general transformation p $\to$ P(q,p,t) so that (q,p) $\to$ (Q,P) will be canonical.
\item Calculate the appropriate generating function (explain why $F_1$(q,Q,t) isn't relevant).
\item Given the new Hamiltonian H'=0. Find the original Hamiltonian.
\item Prove that for $\psi$(q+$\omega$t), the original Hamiltonian is a function of p alone. What does it mean?
\end{enumerate}
\Dn\Dn


%%%%%%%%%%%%%%%%%%%%%%%%%%%%%%
{\bf The solution:}

\Dn
\begin{enumerate}
\item answer 1
\item answer 2
\item answer 3
\item answer 4
%%%%%%%%%%%%%%%%
\item This is how you write an equation
\begin{equation}
\mathcal{I} =\int \limits_0^z  d^4x \sqrt{-g}\left[\varphi^2\left(\mathcal{R}-6sg^{\mu \nu} \kappa_\mu \kappa_\nu\right) + 4 \omega g^{\mu \nu}D_\mu \varphi D_\nu \varphi + \lambda \varphi^4 + \frac{1}{4}g^{\mu \nu} g^{\lambda \sigma}X_{\mu \lambda}X_{\nu \sigma}\right].
\end{equation}

%%%%%%%%%%%%%%%
\item If you wish to add a set of equations (add the ampersands to align the two equations):
\begin{eqnarray}
 s &=& \frac{3+2\omega}{2\omega}\neq 1,\\
 D_\mu \varphi &=& \varphi_{;\mu} + s \kappa_\mu \varphi,
\end{eqnarray}

If you wish to have equations with no numbers add an asterisk
\begin{eqnarray*}
&& g^{\mu \nu} \left( \varphi^2_{;\mu} + 2 \kappa_\mu \varphi^2\right)_{;\nu} = \frac{\partial W_{eff}(\varphi^2)}{\partial \varphi^2}, \\
&& \frac{\partial W_{eff}(\varphi^2)}{\partial \varphi^2} = \frac{1}{3+ 2} \omega \left(\frac{1}{2} \varphi V'(\varphi) - 2V(\varphi)\right).
\end{eqnarray*}
To write a vector
\begin{equation}
\vec{r} = x \hat{x} + y \hat{y} + z\hat{z} \quad ; \quad \vec{r}_i
= \left|\vec{r}\right|\hat{r} \end{equation}
To write a matrix
\begin{equation}
\lambda_1 
= \left( \amatrix { 0 & 1 & 0 \cr 1 & 0 & 0 \cr 0 & 0 & 0}\right), \text{\qquad ``\textbackslash qquad'' makes space in equations and this is how you add text.}
\end{equation}
You can also use ``align'' to tightly align the equal sign, i.e ``='' is aligned: see Eq.~\eqref{eq:align1} and Eq.~\eqref{eq:align2},  also note that {\underline{\bf every end}} of an  equation needs to be punctuated, i.e : ``,'' or ``.'', according to the sentence. 
\begin{align}
\lambda_7 
&= \left( \amatrix { 0 & 0 & 0 \cr 0 & 0 & -i \cr 0 & i & 0}\right), \label{eq:align1}\\
 \lambda_8 
&= \frac{1}{\sqrt{3}}\left( \amatrix { 1 & 0 & 0 \cr 0 & 1 & 0 \cr 0 & -2 & 0}\right),\label{eq:align2}
\end{align}
and centered
\begin{align*}
&&\left[\frac{\lambda_i}{2},\frac{\lambda_j}{2}\right] &= i \displaystyle\sum_{k=1}^{n} f^{ijk}\frac{\lambda_k}{2}, \\
&&f^{147}=f^{165}=f^{246}&=f^{257}=f^{345}=f^{376}= \frac{1}{2}.
\end{align*}

%%%%%%%%%%%%%%%
\item Known arguments are \textit{{\bf not} written in  italic mode}, some have special syntax in LaTeX otherwise just  use ``\textbackslash mathrm'': 
\begin{equation}
\cos (\omega t); \quad  \sin (\omega t);\quad  \mathrm{Tr \left[ \hat A\right] };\quad  \mathrm{det \left[ \hat {\mathcal{A}} \right] }; \quad \tan(\omega t); \quad  \eexp{\frac{t}{\tau}};\quad  \exp{\left(\frac{t}{\tau}\right)}; \quad \log\left(\frac{t}{\tau}\right).
\end{equation}For all Latex related knowledge go to https://en.wikibooks.org/wiki/LaTeX\\
To include a picture use, note that the picture file has to be in the same folder as the *.tex file.

%\includegraphics[scale=0.6]{hp.pdf}
\end{enumerate}

%%%%%%%%%%%%%%%%%%%%%%%%%%%%%%%%%%%%%%%%%%%%%%%%%%%%%%%%%%%%%%%%%%%%%%%%%%%%%%%%%%%%%%%%%%%%
%%%%%%%%%%%%%%%%%%%%%%%%%%%%%%%%%%%%%%%%%%%%%%%%%%%%%%%%%%%%%%%%%%%%%%%%%%%%%%%%%%%%%%%%%%%%
\end{document} 
