\documentclass[11pt,fleqn]{article}


%%%%%%%%%%%%%%%%%%%%%%%%%%%%%%%%%%%%%%%%%
%%% template that does not use Revtex4
%%% but allows special fonts
%%%%%%%%%%%%%%%%%%%%%%%%%%%%%%%%%%%%%%%%%


%%%%%%%%%%%%%%%%%%%%%%%%%%%%%%%%%%%%%%%%%%%
%%%
%%% Please use this template.
%%% Scroll down until you reach the \begin{document} part of the file
%%% Process the file in the Latex site 
%%%
%%%%%%%%%%%%%%%%%%%%%%%%%%%%%%%%%%%%%%%%%%%


% Page setup
\topmargin -1.5cm      
\oddsidemargin -0.04cm   
\evensidemargin -0.04cm  
\textwidth 16.59cm
\textheight 24cm 
\setlength{\parindent}{0cm} 
\setlength{\parskip}{0cm} 


% Fonts
\usepackage{latexsym}
\usepackage{amsmath} 
\usepackage{amssymb} 
\usepackage{bm}
\usepackage{graphicx}


% Math symbols I
\newcommand{\sinc}{\mbox{sinc}}
\newcommand{\const}{\mbox{const}}
\newcommand{\trc}{\mbox{trace}}
\newcommand{\intt}{\int\!\!\!\!\int }
\newcommand{\ointt}{\int\!\!\!\!\int\!\!\!\!\!\circ\ }
\newcommand{\ar}{\mathsf r}
\newcommand{\im}{\mbox{Im}}
\newcommand{\re}{\mbox{Re}}

% Math symbols II
\newcommand{\eexp}{\mbox{e}^}
\newcommand{\bra}{\left\langle}
\newcommand{\ket}{\right\rangle}

% Mass symbol
\newcommand{\mass}{\mathsf{m}} 
\newcommand{\Mass}{\mathsf{M}} 

% More math commands
\newcommand{\tbox}[1]{\mbox{\tiny #1}}
\newcommand{\bmsf}[1]{\bm{\mathsf{#1}}} 
\newcommand{\amatrix}[1]{\begin{matrix} #1 \end{matrix}} 
\newcommand{\pd}[2]{\frac{\partial #1}{\partial #2}}

% Other commands
\newcommand{\hide}[1]{}
\newcommand{\drawline}{\begin{picture}(500,1)\line(1,0){500}\end{picture}}
\newcommand{\bitem}{$\bullet$ \ \ \ }
\newcommand{\Cn}[1]{\begin{center} #1 \end{center}}
\newcommand{\mpg}[2][1.0\hsize]{\begin{minipage}[b]{#1}{#2}\end{minipage}}
\newcommand{\Dn}{\vspace*{3mm}}

% Figures
\newcommand{\putgraph}[2][0.30\hsize]{\includegraphics[width=#1]{#2}}

% heading
\newcommand{\exnumber}[1]{\newcommand{\exnum}{#1}}
\newcommand{\heading}[1]{\begin{center} {\Large {\bf Ex\exnum:} #1} \end{center}}
\newcommand{\auname}[1]{\begin{center} {\bf Submitted by:} #1 \end{center}}


\begin{document} %This is where you start editing things%
%%%%%%%%%%%%%%%%%%%%%%%%%%%%%%%%%%%%%%%%%%%%%%%%%%%%%%%%%%%%%%%%%%%%%%%%%%%%%%%%%%%%%%%%%%%%
%%%%%%%%%%%%%%%%%%%%%%%%%%%%%%%%%%%%%%%%%%%%%%%%%%%%%%%%%%%%%%%%%%%%%%%%%%%%%%%%%%%%%%%%%%%%
%%%%%%%%%%%%%%%%%%%%%%%%%%%%%%%%%%%%%%%%%%%%%%%%%%%%%%%%%%%%%%%%%%%%%%%%%%%%%%%%%%%%%%%%%%%%
%%%%%%%%%%%%%%%%%%%%%%%%%%%%%%%%%%%%%%%%%%%%%%%%%%%%%%%%%%%%%%%%%%%%%%%%%%%%%%%%%%%%%%%%%%%%

\exnumber{0909A3}
\heading{The generalized canonical transformation} 
\auname{Michael Pukshanski, Lior Weitzhandler}


{\bf The problem:}
\Dn
A system is described by the Hamiltonian H(q,p,t). The coordinate q is transformed q $\to$ Q=$\psi$(q,t)
\begin{enumerate}
\item Find the most general transformation p $\to$ P(q,p,t) so that (q,p) $\to$ (Q,P) will be canonical.
\item Calculate the appropriate generating function (explain why $F_1$(q,Q,t) isn't relevant).
\item Given the new Hamiltonian H'=0. Find the original Hamiltonian.
\item Prove that for $\psi$(q+$\omega$t), the original Hamiltonian is a function that is linear to p.
\end{enumerate}
\Dn\Dn


%%%%%%%%%%%%%%%%%%%%%%%%%%%%%%
{\bf The solution:}

\Dn
\begin{enumerate}
\item We will derive here the transformation for P:
\begin{equation}
\left[Q,P\right]=\frac{\partial Q}{\partial q}\cdot\frac{\partial P}{\partial p}-\frac{\partial Q}{\partial p}\frac{\partial P}{\partial q}=\frac{\partial Q}{\partial q}\cdot\frac{\partial P}{\partial p}=\frac{\partial \psi(q,t)}{\partial q}\cdot\frac{\partial P}{\partial p}=1
\end{equation}
\begin{equation}
\frac{\partial P}{\partial p}=\frac{\partial q}{\partial \psi}=\frac{\partial q}{\partial Q}
\end{equation}
\begin{equation}
\Rightarrow 
\boxed {P=p\cdot\frac{\partial q}{\partial Q}+f(q,t)}
\end{equation}
\item Finding the generating function:
\newline
We will choose the third class of generating function $F_{3}$(p,Q,t). For that we will define q as a function of  Q and t, leading to q=g(Q,t), because it is given that Q=$\psi$(q,t). Consequently f can be expressed with Q and t.
\begin{equation}
-P=\frac{\partial F_{3}}{\partial Q}
\end{equation}
\begin{equation}
-q=\frac{\partial F_{3}}{\partial p}
\end{equation}
\newline
\begin{equation}
\frac{\partial F_{3}}{\partial Q}=-p\cdot\frac{\partial q}{\partial Q}-f(q,t)
\end{equation}
\begin{equation}
F_{3}=-pq- \int_{}^{}f(Q,t)dQ+h(p,t)
\end{equation}
\begin{equation}
\frac{\partial F_{3}}{\partial p}=-q+\frac{\partial h}{\partial p}=-q \; \Rightarrow\; h=l(t)
\end{equation}
\begin{equation}
F_{3}=-pq- \int_{}^{}f(Q,t)dQ+l(t) 
\end{equation}
\begin{equation}
\Rightarrow \boxed{F_{3}=-p\cdot g(Q,t)-\int_{}^{} f(Q,t)dQ}
\end{equation}
We ignored the function that depends only on time because it can't affect the equation of motion.
\newline
The reason $F_1$(q,Q,t) isn't relevant is because it produces p and P from its partial derivatives and we want to use the given transformation q $\to$ Q=$\psi$(q,t). Also, we can see in equation  (8) that if we change there the derivative to q (which will make the generating function be $F_1$) we will get an expression that contains p, but $F_1$ is not depended on p, so we will have a contradiction.
\item Here we will find the original Hamiltonian.
\newline
we know that:
\begin{equation}
0=\grave{\mathcal{H}}=\mathcal{H}+\frac{\partial F_{3}}{\partial t}
\end{equation}
\begin{equation}
\mathcal{H}=-\frac{\partial F_{3}}{\partial t}
\end{equation}
\begin{equation}
\Rightarrow\boxed{\mathcal{H}=\frac{\partial (p\cdot g(Q,t))}{\partial t}+\frac{\partial (\int_{}^{}f(g(q,t),t)dQ)}{\partial t}}
\end{equation}
\item Now it is given that q $\to$ Q=$\psi$(q+$\omega$t). We can derive from that the new connection between q and Q:
\begin{equation}
q=u(Q)-\omega t
\end{equation}
We shall find the original Hamiltonian.
\begin{equation}
\frac{\partial F_{3}}{\partial Q}=-p\cdot\frac{\partial q}{\partial Q}-f(Q,t)=-p\cdot \frac{\partial u}{\partial Q} -f(Q,t)
\end{equation}
\begin{equation}
F_{3}=-p\cdot u(Q)-\int_{}^{} f(Q,t)dQ+h(p,t)
\end{equation}
\begin{equation}
\frac{\partial F_{3}}{\partial p}=-u(Q)+\frac{\partial h}{\partial p}=-q=-u(Q)+\omega t \; \Rightarrow\; h(p.t)=\omega t \cdot p
\end{equation}
\begin{equation}
F_{3}=-p\cdot u(Q)-\int_{}^{} f(Q,t)dQ+ \omega t \cdot p
\end{equation}
\begin{equation}
\mathcal{H}=-\frac{\partial F_{3}}{\partial t}
\end{equation}
\begin{equation}
\Rightarrow\boxed{\mathcal{H}=\omega \cdot p+ \frac{\partial (\int_{}^{}f(Q,t)dQ)}{\partial t}}
\end{equation}
The original Hamiltonian is linear to p as requested.
%\includegraphics[scale=0.6]{hp.pdf}
\end{enumerate}

%%%%%%%%%%%%%%%%%%%%%%%%%%%%%%%%%%%%%%%%%%%%%%%%%%%%%%%%%%%%%%%%%%%%%%%%%%%%%%%%%%%%%%%%%%%%
%%%%%%%%%%%%%%%%%%%%%%%%%%%%%%%%%%%%%%%%%%%%%%%%%%%%%%%%%%%%%%%%%%%%%%%%%%%%%%%%%%%%%%%%%%%%
\end{document} 
